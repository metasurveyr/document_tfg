Con profundo agradecimiento y emoción, quiero comenzar reconociendo a mi tutora, Natalia da Silva. Desde la clase de Nuevas Tecnologías para el Análisis de Datos, ella despertó en mí una pasión por el aprendizaje continuo y me motivó a explorar la importancia del uso y desarrollo de herramientas computacionales en la profesión de un estadístico. Además, Natalia siempre ha sido para mí una referente en lo que respecta al aprendizaje estadístico y el desarrollo de la comunidad de R, demostrando su compromiso con la innovación y el crecimiento de nuestra disciplina. Su apoyo incondicional, paciencia y disposición para guiarme en cada etapa de este proyecto fueron fundamentales, así como su confianza en mis capacidades. Más adelante, tuve la oportunidad de ser ayudante en la nueva versión llamada Ciencia de Datos con R, lo que me permitió continuar creciendo en este apasionante campo.

Quiero extender mi gratitud a todos los profesores de la Licenciatura en Estadística, quienes me brindaron las herramientas necesarias para llevar a cabo este trabajo. En particular, agradezco a Alejandra Marroig, cuyas ideas y consejos enriquecieron este proyecto al vincularlo con casos de uso de encuestas continuas de hogares. También a Ramón Álvarez, quien confió en mis capacidades desde el principio, permitiéndome aplicar herramientas computacionales que no solo alimentaron este proyecto, sino que también me dieron la oportunidad de compartir ese conocimiento en charlas internas en el IESTA.

A mis compañeros de la Licenciatura en Estadística, quiero agradecerles por los momentos compartidos a lo largo de esta carrera. Juntos atravesamos largas jornadas de estudio, debates y aprendizaje mutuo, mientras este proyecto comenzaba a tomar forma. En especial, quiero destacar a mis amigos Ana Vignolo y Maximiliano Saldaña, quienes siempre estuvieron allí para escucharme, brindarme su apoyo y ayudarme a mantener el enfoque en los momentos más desafiantes.

A mis compañeros de trabajo en Cognus, les agradezco profundamente por su comprensión y respaldo, permitiéndome equilibrar mis responsabilidades laborales con la dedicación necesaria para este proyecto. Su apoyo y ánimo fueron esenciales para alcanzar este objetivo.

No puedo dejar de mencionar a mis amigos, quienes contribuyeron desde sus propias perspectivas. A Fabricio Machado, por sus ideas, consejos y la posibilidad de probar ejemplos reales basados en encuestas continuas de hogares; a Matías Sesser, quien, aunque nunca entendió del todo qué trataba este proyecto, siempre me ofreció su apoyo incondicional y valiosos consejos sobre herramientas computacionales; y a Giannina, quien siempre estuvo dispuesta a escucharme en los momentos más difíciles, cuando la desmotivación me afectaba, dándome ánimo para seguir adelante. Su paciencia y comprensión significaron mucho para mí.

También quiero expresar mi más profundo agradecimiento a mis alumnos, quienes han sido una constante fuente de inspiración. Sus preguntas, curiosidad y entusiasmo me recordaron la importancia de seguir aprendiendo y compartiendo conocimientos. Cada clase fue una oportunidad de crecimiento mutuo, y su energía me impulsó a dar lo mejor de mí. Este trabajo también es para ellos, porque fueron y seguirán siendo una parte fundamental de mi camino.

Finalmente, quiero agradecer profundamente a mi familia, quienes siempre me brindaron su apoyo constante. A mi padre, Oscar, y a su esposa, Gabriela, quienes siempre me dieron confianza, aliento y amor incondicional. También a mis hermanos y sobrinos, quienes llenaron mi camino de cariño y energía, aunque debo admitir que, como el resto de la familia, nunca terminaron de entender del todo qué hace exactamente un estadístico. Sin embargo, su apoyo y celebraciones en cada pequeño avance hicieron toda la diferencia. Y de manera muy especial, a mi sobrina Florencia, quien siempre estuvo dispuesta a escucharme en mis momentos de duda y frustración, dándome ánimo y fuerza para seguir adelante. Este logro t   ambién es de todos ellos.

A todos, gracias por ser parte de este viaje tan significativo en mi vida.
